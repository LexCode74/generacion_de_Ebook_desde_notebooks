% --- Preámbulo compartido ---
% OJO: babel se carga en master.tex con [spanish,es-noshorthands]

% Matemáticas
\usepackage{amsmath,amssymb,amsthm}
\usepackage{mathtools}      % mejoras sobre amsmath
\usepackage{bm}             % \bm para negritas en matemáticas
\usepackage{cases}          % entorno numcases

% Gráficos y color
\usepackage{graphicx,xcolor}

% Página
\usepackage{geometry}
\geometry{margin=2.5cm}

% Profundidad de numeración y de índice
% 0=part, 1=chapter, 2=section, 3=subsection, 4=subsubsection
\setcounter{secnumdepth}{3}
\setcounter{tocdepth}{3}

% Numeración de ecuaciones por capítulo
\numberwithin{equation}{chapter}

% Tipografía de párrafo (más tolerante)
\setlength{\parskip}{6pt}
\setlength{\parindent}{0pt}
\setlength\emergencystretch{3em}
\sloppy

% --- Captions y tipo 'none' para longtable ---
\usepackage{caption}
\usepackage{ltcaption} % longtable + caption

% Declaramos un tipo de caption 'none' (sin etiqueta ni contador)
\DeclareCaptionType{none}[][]
\captionsetup[none]{labelformat=empty,labelsep=none}

% Código fuente (listings; sin minted)
\usepackage{listings}
\lstset{
  language=Python,
  basicstyle=\ttfamily\small,
  keywordstyle=\color{blue},
  commentstyle=\color{green!50!black},
  stringstyle=\color{red!70!black},
  frame=single,
  breaklines=true,
  upquote=true,
  showstringspaces=false,
  columns=fullflexible
}

% (Opcional) acentos correctos en listings:
% \lstset{
%   literate=
%     {á}{{\'a}}1 {é}{{\'e}}1 {í}{{\'i}}1 {ó}{{\'o}}1 {ú}{{\'u}}1
%     {Á}{{\'A}}1 {É}{{\'E}}1 {Í}{{\'I}}1 {Ó}{{\'O}}1 {Ú}{{\'U}}1
%     {ñ}{{\~n}}1 {Ñ}{{\~N}}1
% }

% --- Tipografía monoespaciada con buen soporte Unicode ---
\setmonofont[Scale=MatchLowercase]{DejaVu Sans Mono}

% Entornos / teoremas
\newtheorem{definicion}{Definición}[chapter]
\newenvironment{ejercicio}{\par\medskip\noindent\textbf{Ejercicio.}}{\par\medskip}

% Atajos
\newcommand{\vect}[1]{\bm{#1}}
\newcommand{\R}{\mathbb{R}}
\newcommand{\N}{\mathbb{N}}
\newcommand{\Z}{\mathbb{Z}}

% --- Ajustes de títulos (evita saltos de nivel y mejora cabeceras) ---
\usepackage{titlesec}
% Estilo del capítulo (texto grande, con "Capítulo N.")
\titleformat{\chapter}[display]
  {\bfseries\LARGE}
  {\chaptername\ \thechapter.}{1ex}{}
\titlespacing*{\chapter}{0pt}{-0.5ex}{2.5ex}

% Asegurar numeración y consistencia de niveles para Hyperref
\usepackage{bookmark} % mejora marcadores PDF
% Si alguna celda hubiese generado \section* accidentalmente y luego \subsection,
% Hyperref se queja. Titlesec ya ayuda, pero podemos reiniciar contadores al subir nivel:
\usepackage{etoolbox}
\pretocmd{\section}{\setcounter{subsection}{0}\setcounter{subsubsection}{0}}{}{}
\pretocmd{\subsection}{\setcounter{subsubsection}{0}}{}{}

% ==== IMÁGENES (persistente: vive en estilo.tex) ====
\usepackage{graphicx}
\usepackage{grffile}
\usepackage{float} % por si usamos [H] en figuras
\DeclareGraphicsExtensions{.pdf,.png,.jpg,.jpeg}

% Búsqueda de imágenes: rutas vistas desde build/
\makeatletter
\newcommand*\addpicpath{%
  \graphicspath{{./}{./capitulo6_files/}{./capitulo7_files/}{./capitulo8_files/}{./capitulo9_files/}{./apendice_files/}{../notebooks/}}%
}
% Llamamos siempre que empiece el documento
\AtBeginDocument{\addpicpath}
\makeatother

% Escala por defecto (si no se indica width en cada imagen)
\setkeys{Gin}{keepaspectratio,width=\linewidth}

% Atajos para usar desde Markdown:
% \img[width=.4\textwidth]{archivo.png}
\newcommand{\img}[2][]{\begin{center}\includegraphics[#1]{#2}\end{center}}

% \Fig{archivo.png}{.4\textwidth}{Título de la figura}
\newcommand{\Fig}[3]{%
\begin{figure}[H]\centering
\includegraphics[width=#2]{#1}
\caption{#3}
\end{figure}}
% ==== FIN IMÁGENES ====
% ==== Portada y licencia (persistentes desde estilo.tex) ====
\usepackage{hyperref}
\urlstyle{same}
\usepackage{xstring} % para limpiar "Autor " si viene en \author

% Pie de página global (si ya lo tienes configurado, deja lo tuyo)
\usepackage{fancyhdr}
\pagestyle{fancy}
\fancyhf{}
\fancyfoot[L]{\small Copyright EDS e IA 2025.}
\fancyfoot[R]{\small \url{ecuaciondiferencialejerciciosresueltos.com}}
\renewcommand{\headrulewidth}{0pt}
\renewcommand{\footrulewidth}{0pt}

% Edición (cámbiala aquí si deseas otra leyenda)
\newcommand{\bookedition}{PRIMERA EDICIÓN}

% No mostrar fecha nunca en la portada
\makeatletter
\renewcommand{\@date}{} % vacía por si master pone \date{\today}

% \maketitle personalizado:
\renewcommand{\maketitle}{%
  \begingroup
    \thispagestyle{fancy} % pie también en portada
    \begin{titlepage}
      \centering
      % Título en negritas (grande)
      {\bfseries\LARGE \@title\par}
      \vspace{1.8cm}
      % Rotulo "Autor" en una línea y el nombre DEBAJO en negritas
      {\Large \textbf{Autor}\par}
      \vspace{0.6cm}
      % Limpiar si master trae "Autor <nombre>"
      \def\AuthorClean{\@author}%
      \IfBeginWith{\AuthorClean}{Autor }{%
        \StrBehind{\AuthorClean}{Autor }[\AuthorClean]%
      }{}%
      {\Large \textbf{\AuthorClean}\par}
      \vspace{1.2cm}
      % Edición debajo del nombre
      {\large \bookedition\par}
      \vfill
    \end{titlepage}
  \endgroup
  % Página siguiente: Licencia Creative Commons
  \clearpage
  \thispagestyle{fancy}
  \null\vfill
  \begin{center}
    {\small
      Este material se distribuye bajo la licencia\\[0.3em]
      \textbf{Creative Commons Atribución-NoComercial-SinDerivadas 4.0 Internacional}\\[0.6em]
      \href{https://creativecommons.org/licenses/by-nc-nd/4.0/deed.en}{https://creativecommons.org/licenses/by-nc-nd/4.0/deed.en}
    }
  \end{center}
  \vfill\null
  \clearpage
}
\makeatother
% ==== FIN portada y licencia ====



